\section{Introduction}
In this appendix is descried the required procedure to build the benchmark project that contains the YCSB adapters.
Are then shown the required commands to be executed in order to execute the benchmarks and the available values to be configured for each of the supported YCSB adapter.

\section{Preliminary operations}
In order to build the benchmark project, available at \url{https://github.com/Arci/kundera-benchmark}, some libraries needs to be downloaded since are not available in any maven repository:
\begin{itemize}
\item Azure Table extension\\ \url{https://github.com/Arci/kundera-azure-table}
\item GAE Datastore extension \\\url{https://github.com/Arci/kundera-gae-datastore}
\end{itemize}

\noindent The Azure Table extension tests requires to run a reachable storage emulator on Windows so if you do not want to execute test while build run \texttt{mvn clean install -DskipTests}.
 
\noindent Test for the Datastore extension can be executed without any configuration as they're executed thought google in-memory Datastore stub.

\noindent Since also YCSB is not available in any maven repository, it must be downloaded (\url{https://github.com/brianfrankcooper/YCSB/}) and installed locally, always through \texttt{mvn install}.

\newparagraph Now all the required dependency for kundera-benchmark should be resolved so is possible to install it with maven:
\texttt{mvn clean install}
then lunch the command:
\texttt{mvn dependency:copy-dependencies}
this will create a directory called \texttt{dependency} in the target directory containing all the jars of the dependencies.

\section{Run tests for low-level API version}
\label{appendix:ycsb-low-level}
To run the benchmarks, after having built the project, the two phases of the YCSB tests can be executed through the command:

\begin{verbatim}
java -cp KUNDERA-BENCHMARK-JAR-LOCATION:PATH-TO-DEPENDENCY-FOLDER/*
com.yahoo.ycsb.Client -t -db DATABASE-ADAPTER-CLASS-TO-USE
-P PATH-TO-WORKLOAD -P PATH-TO-PROPERTY-FILE 
-s -threads THREAD-TO-USE -PHASE > OUTPUT_FILE
\end{verbatim}

\noindent where PHASE should be \texttt{load} for \textbf{load} phase or \texttt{t} for \textbf{transaction} phase.

\newparagraph  Available adapter classes are:
\begin{itemize}
\item \texttt{it.polimi.ycsb.database.AzureTableClient} for Azure Table
\item \texttt{it.polimi.ycsb.database.DatastoreClient} for GAE Datastore
\item \texttt{it.polimi.ycsb.database.KunderaHBaseClient} for Hbase
\end{itemize}

\subsection{Property files}
As can be seen from the command, a property file must be specified. Property files must provide information to locate the database to test when running the benchmarks on the low-level API versions.

\paragraph{Google Datastore} The available properties are:
\begin{itemize}
\item \texttt{url} \textit{required}
\item \texttt{port} \textit{optional}, default is 443
\item \texttt{username} \textit{required}, the username of an admin on the remote application
\item \texttt{password} \textit{required}, can be omitted if tests are against localhost
\end{itemize}

\paragraph{Azure Table} The available properties are:
\begin{itemize}
\item \texttt{emulator} [true|false]
\item \texttt{account.name} \textit{required} if not using emulator, available from azure portal
\item \texttt{account.key} \textit{required} if not using emulator, available from azure portal
\item \texttt{protocol} [http|https] \textit{optional}, default is https
\end{itemize}
\noindent If \texttt{emulator} is set to textit{true} the remaining properties are ignored.

\paragraph{Hbase}
The properties must be configured inside the file because, to be more accurate w.r.t. the Kundera client, connection cannot be done in the \texttt{init()} method. 

\noindent The properties can be set modifying the following constants:
\begin{itemize}
\item \texttt{node}, the master node location
\item \texttt{port}, the master node port
\item \texttt{zookeeper.node}, the node location for \texttt{hbase.zookeeper.quorum}
\item \texttt{zookeeper.port}, the node port for \texttt{hbase.zookeeper.property.clientPort}
\end{itemize}

\noindent Since property file is not needed for Hbase, it does not need to be specified while running the benchmarks.

\section{Run tests for Kundera version}
\label{appendix:ycsb-kundera}
To run the benchmarks, after having built the project, the two phases of the YCSB tests can be executed through the command:

\begin{verbatim}
java -cp KUNDERA-BENCHMARK-JAR-LOCATION:PATH-TO-DEPENDENCY-FOLDER/*
com.yahoo.ycsb.Client -t -db DATABASE-ADAPTER-CLASS-TO-USE
-P PATH-TO-WORKLOAD -s -threads THREAD-TO-USE -PHASE > OUTPUT_FILE
\end{verbatim}

\noindent where PHASE should be \texttt{load} for \textbf{load} phase or \texttt{t} for \textbf{transaction} phase.

\newparagraph Available adapter classes are:
\begin{itemize}
\item \texttt{it.polimi.ycsb.database.KunderaAzureTableClient} for kundera-azure-table extension
\item \texttt{it.polimi.ycsb.database.KunderaDatastoreClient} for kundera-gae-datastore extension
 \item\texttt{it.polimi.ycsb.database.KunderaHBaseClient} for kundera-hbase extension
\end{itemize}

\subsection{\textit{persistence.xml} configuration}
In the \textit{persistence.xml} file, each persistence unit must be configured to locate the database to test.

\noindent Configuration possibilities are described in the appendix \ref{app:kconfig}.

\noindent Hbase configuration make use also of a datastore specific property file \texttt{hbase-properties.xml} in which can be configured the value for \texttt{hbase.zookeeper.quorum} and \texttt{hbase.zookeeper.property.clientPort}. If those values needs to be changed can be configured in the \texttt{hbase-properties.xml} file.