\section{Introduction}
In this chapter we expose the motivations that lead us to conduct this work, in particular, we analyze the current problems in the NoSQL service implementation of the CPIM library and propose a solution to address them and, at the same time, increasing the number of NoSQL database supported by the library.
\noindent Furthermore, we will discuss why we decided to include the possibility for the CPIM library users to be able to migrate and synchronize data, across databases, by means of a migration system called \textit{Hegira}.

\section{CPIM NoSQL service}
The CPIM library uses various implementation of the JPA interface to ease the communication with different NoSQL databases:
\begin{itemize}
\item Google Datastore is supported by means of the Google JPA implementation around Datastore API;
\item Azure Tables is supported through \textit{jpa4azure} a third party implementation of the JPA interface for Tables;
\item Amazon Simple DB is supported through \textit{simpleJPA} a third party implementation of the JPA interface for Simple DB.
\end{itemize}
\noindent By choosing the cloud provider inside the \textit{configuration.xml} the library knows, at run-time, which interface should be used for the service and this holds also for the NoSQL service. Hence to use Google Datastore as NoSQL database, Google has to be selected as cloud provider.

\noindent The aim of CPIM is to offer to the user a way of writing cloud application in a provider-independent fashion to be able to migrate the application from a provider to another without the necessity of re-engineer the application. For the NoSQL service this is achieved by meas of the JPA interface that, beside is not a standard for accessing NoSQLs, many projects came into play trying to bring the benefit of the JPA interface also in the NoSQL world.

\newparagraph As discussed in chapter \ref{chap:sota}, the use of the JPA interface in defining a common interface for accessing NoSQL databases is a solution widely adopted, thus the choice of using the JPA as abstraction layer for the NoSQL service in the CPIM is a valuable choice. 
However the current implementation have significant problem:
\begin{enumerate}
\item[\textbf{P.1}] the application code written to interact with the NoSQL service is not interoperable and thus require the user to modify the application code in order to move to a different cloud provider.  
\end{enumerate}
\noindent Moreover the NoSQL service suffer of some limitations too:
\begin{enumerate}
\item[\textbf{L.1}] the choice of the NoSQL database is strictly bind to the selected cloud provider; 
\item[\textbf{L.2}] even if the selection of the NoSQL database would be possible, the number of supported NoSQL database is very limited.
\end{enumerate}

\noindent For \textbf{P.1}, the problem reside in the fact that for each of the currently supported database, has been found and integrated into CPIM, a specific implementation of the JPA interface. Even through JPA is a well defined standard, not every JPA provider follows strictly the specification and thus different provider can behave differently while persisting the same entities since they interpret differently the semantic of some JPA annotation.

\noindent An example of this problem is how Collection fields are currently handled in CPIM. In the Google JPA implementation for Datastore and in the JPA implementation for Amazon SimpleDB, Collection fields are handled correctly, with respect to the JPA specification, thought the \texttt{@ElementCollection} annotation, while, in the JPA implementation for Azure Tables, Collection fields needs to be annotated with the \texttt{@Embedded} annotation. This require a modification of the code and thus eliminates the effort of CPIM in achieving code portability among PaaS.

\newparagraph As regards \textbf{L.1} and \textbf{L.2}, we would give to the user the ability to persist data in the database that best fit his requirements. For example if the user application will generate data that should be processed with Hadoop, the best solution is to store those data in an HBase instance since its integrate easily in Hadoop. Therefore we want to make the user able to persist different entities in different datastore based on his needs and without the limitation of a specific NoSQL technology.

\subsection{Proposed solution}
The proposed solution is mainly about the integration of Kundera, a JPA compliant ORM for NoSQL databases, as unique persistence layer for the NoSQL service. This integration will be useful to solve the problems and mitigte the limitation outlined as follow:
\begin{itemize}
\item since Kundera will be the unique persistence provider for the library we will relay only on one implementation of the JPA interface overtaking the problem \textbf{P.1}, related to different interpretation of the JPA annotation, and thus achieving complete portability of the code of the model since no modifications are required to work with different NoSQL database through Kundera;
\item the integration of Kundera permits a redesign of the NoSQL service aimed to decouple the chosen PaaS provider and the NoSQL technology overcoming limitation \textbf{L.1} by giving to the user the ability to decide which technology is more suitable for his needs. Furthermore exploiting the polyglot-persistence provided by Kundera, the user will be able to persist entities within different NoSQL databases at the same time simply by define accordingly the persistence unit in the \textit{persistence.xml} file;
\item choosing Kundera as persistence layer we can actually take advantage of the already developed extension for many different NoSQL databases, adding as a result the support of those database to CPIM, and thus overcoming the limitation \textbf{L.2};
\end{itemize}

\noindent There are many reasons why we choose to use Kundera as persistence provider for the NoSQL service of the CPIM library. The main reason is that Kundera, through the use of the JPA interface will permit to the user to handle the complexity of NoSQL databases with expertise he already uses for SQL systems. Furthermore Kundera is in the field from 2010 and thus have a big and active community, built in many years of activity, and has been used successfully in some production environment.

\noindent The only drawback is that Kundera does not support any of the NoSQL datastore currently supported by CPIM. Fortunately Kundera have, as its primary goal, to make the library as much extensible as possible to let developers build their own client around new NoSQL technologies. The solution will thus be to develop the needed extensions for Kundera.

\newparagraph The work on the CPIM NoSQL service will thus require:
\begin{itemize}
\item the integration of Kundera as the unique persistence provider in the NoSQL service of CPIM;
\item the development of two brand new Kundera extensions, one for Google Datastore and one for Azure Tables.
\end{itemize}

\section{Hegira integration}
NoSQL technologies do not offer a common querying language to interact with them, as SQL does for RDBMS. Furthermore, NoSQLs offer a simpler interface with respect to RDBMS. This require user to interact with NoSQL databases at a lower level of interaction, moreover no effort is made by NoSQL database in hiding to the user the physical representation of the data, this move a good amount of developing effort toward the user.
This approach, and the many different NoSQL solution available nowadays, causes a lock-in on the technology chosen since re-engineer an application is often an operation with huge costs for companies.

\newparagraph To mitigate this lock-in problem we want to extend CPIM to make it able to interact with \textit{Hegira} a migration system able to perform interoperable data migration and synchronization across column-based NoSQL databases \cite{paper:modaclouds-deliverable}.

\noindent \textit{Hegira} is already able to migrate data offline but in many cases this solution is not acceptable since this require to turn off the application for a period of time that depends on how many data needs to be migrated to the new database. 
Downtime cost and risk of data loss can be problematic so \textit{Hegira} was extended to be able to perform a live-migration of the data by keeping them synchronized on the source and the destination database.
This feature needs to be exploited at application level and thus we decide to embed it inside the CPIM NoSQL service in order to make it as transparent as possible to the user.

\noindent CPIM needs to be aware of the state of both the synchronization and migration systems and acts accordingly intercepting user operation and sending data manipulation queries (DMQ) to the migration system wile keeping the data synchronized interacting with the synchronization system when required.