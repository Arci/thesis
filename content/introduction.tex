Introduzione al lavoro. Inizia direttamente, senza nessuna sezione.

\noindent Argomenti trattati suddivisi sezione per sezione\dots

\section*{Original Contributions}
This work include the following original contributions:
\begin{itemize}
\item \dots riassunto sintetico dei diversi contributi
\item \dots
\item \dots
\end{itemize}

\section*{Outline of the Thesis}
This thesis is organized as follows: 
\begin{itemize}
\item In Chapter \ref{chap:sota} is described the current evolution of NoSQL databases . As a first introduction is discussed why in this years this technology have emerged over SQL solutions and what are the main differences among those technology, the second part aims to underline the lack of a common language for NoSQLs in contrast to SQL-99 for SQL databases.
\item In Chapter \ref{chap:ps} \dots
\item In Chapter \ref{chap:kundera} is dedicated to the develop of the two Kundera client extension that have been developed in order to support Google Datastore and Azure Tables that will be then used in CPIM as adapters for the relative database.
\item In Chapter \ref{chap:cpim} is presented the work made on CPIM. As a first step is described a modification in the CPIM NoSQL service aimed to integrate Kundera as unique persistence layer for NoSQL access using the standard JPA interface, the library was previously using several different JPA implementation one for each of the supported databases. Furthermore is discussed the extension of CPIM to include an interaction with the migration system \textit{Hegira}.
\item In Chapter \ref{chap:eval} \dots
\item In Chapter \ref{chap:conclusions} draws the conclusions on the entire work and proposes some possible future works.
\end{itemize}

