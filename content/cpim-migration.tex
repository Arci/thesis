\section{Introduction}
In this appendix is presented the new configuration file added to CPIM to support the configuration of the migration system.

\section{\textit{migration.xml}}
The skeleton of the \textit{migration.xml} file is the following:

\begin{verbatim}
<?xml version="1.0" encoding="UTF-8" standalone="no"?>
<migration>
    <zooKeeper>
        <type>...</type>
        <connection>...</connection>
        <range>...</range>
    </zooKeeper>
    <backup>
        <execute>...</execute>
        <type>...</type>
        <directory>...</directory>
        <prefix>...</prefix>
    </backup>
    <followCascades>...</followCascades>
</migration>
\end{verbatim}

\noindent Three are the main section that can be configured:
\begin{enumerate}
\item \textbf{zooKeeper client}
\item \textbf{sequence number backup}
\item \textbf{follow cascades while build statements}
\end{enumerate}

The first two options are the most complex and are described in the following sections, the third option can assume the values of \textit{true} or \textit{false} but is \textit{optional} since is set to \textbf{false} by default.
In case the value is set to \textit{true}, the statement builders, when builds the statements from objects, will read the values for the \texttt{@CascadeType} annotation and, if necessary, builds the cascade statements and sends them to \textit{Hegira} in the correct execution order as described in \ref{sec:statements} 

\subsection{Configure the zooKeeper client}

\subsection{Configure a sequence number backup}
A sequence number backup can be configured either on a blob storage (when running on PaaS) or to file (when running to IaaS).

\noindent The \texttt{<execute>} tag define if backup should or should not be performed, the possible values are \textit{yes} or \textit{no}. This tag is \textit{optional} since is set as default to \textbf{yes}.

\newparagraph For the other configuration options, two cases must be distinguished:

\paragraph{Backup to Blob} An example configuration would be the following one:
\begin{verbatim}
<backup>
    <type>blob</type>
    <prefix>SeqNumber_</prefix>
</backup>
\end{verbatim}

\noindent The text in the \texttt{<prefix>} tag will be prefixed to each blob created to avoid file name conflicts.
This prefix is an \textit{optional} field since as default is set to \textbf{SeqNumber\textunderscore}.

\paragraph{Backup to file} An example configuration would be the following one:
\begin{verbatim}
<backup>
    <type>file</type>
    <directory>/backups</directory>
    <prefix>SeqNumber_</prefix>
</backup>
\end{verbatim}

\noindent The \texttt{<directory>} tag is mandatory and must contain the path to the directory in which the backup file will be placed and retrieved in case they needs to be restored.

\noindent For the \texttt{<prefix>} tag, the same considerations made for backup to blob applies.

\section{Use CPIM without migration system}
The \textit{migration.xml} file is not necessary if the user won't use the migration system. If the file is not present inside the \texttt{META-INF} folder, the NoSQL service will interact only with the underlying persistence provider without instantiating any of the required classes.

\noindent This is possible due to the lazy initialization of those components that are initialized only the first time are actually used since are built with a singleton pattern.