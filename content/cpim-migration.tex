\section{Introduction}
In this appendix is presented the new configuration file added to CPIM to support the various configurations available for the interaction with the migration system.

\section{\textit{migration.xml}}
The template of the \textit{migration.xml} file is the following:

\begin{lstlisting}[language=XML, caption=migration.xml template]
<?xml version="1.0" encoding="UTF-8" standalone="no"?>
<migration>
    <zooKeeper>
        <type>...</type>
        <connection>...</connection>
        <range>...</range>
    </zooKeeper>
    <backup>
        <execute>...</execute>
        <type>...</type>
        <directory>...</directory>
        <prefix>...</prefix>
    </backup>
    <followCascades>...</followCascades>
</migration>
\end{lstlisting}

\newparagraph Three are the main section that can be configured:
\begin{enumerate}
\item ZooKeeper client;
\item sequence number backup;
\item follow cascades while build statements.
\end{enumerate}

\noindent The first two options are the most complex and are described in the following sections, the third option can assume be \textit{true} or \textit{false} but is \textit{optional} since is set to \textbf{false} by default.
In case the value is set to \textit{true}, the statement builders when builds the statements from objects will read the values specified for the \texttt{@CascadeType} annotation and, if necessary, builds the cascade statements and sends them to \textit{Hegira} in the correct execution order as described in \ref{sec:statements}.

\subsection{Configure the ZooKeeper client}
For ZooKeeper client, must be chosen the \textit{thread} or the \textit{http} version as described in \ref{sec:sync}.

\paragraph{Thread version} An example configuration would be the following one:
\begin{lstlisting}[language=XML, caption=ZooKeeper - thread type configuration]
<zooKeeper>
    <type>thread</type>
    <connection>localhost:2181</connection>
    <range>50</range>
</zooKeeper>
\end{lstlisting}

\noindent The \texttt{<connection>} tag is \textbf{required} and should contains an address in the form \texttt{host:port} which should be the host address and the port on which the ZooKeeper service is running.

\noindent The \texttt{<range>} is \textit{optional} since the default is \textbf{10} and is the default dimension of the range that \texttt{SeqNumberDispenser}(s) use to ask sequence numbers to the synchronization system.

\paragraph{HTTP version} An example configuration would be the following one:
\begin{lstlisting}[language=XML, caption=ZooKeeper - http type configuration]
<zooKeeper>
    <type>http</type>
    <connection>http://server.com/hegira-api/zkService</connection>
    <range>50</range>
</zooKeeper>
\end{lstlisting}

\noindent In this case the \texttt{<connection>} tag should contains the API base path of the server in which the service is running.

\noindent For the \texttt{<range>} tag the same consideration made for the thread case applies.
 
\subsection{Configure a sequence number backup}
A sequence number backup can be configured either on a blob storage (when running on PaaS) or to file (when running on IaaS).

\newparagraph  The \texttt{<execute>} tag define if backup should or should not be performed, the possible values are \textit{yes} or \textit{no}. This tag is \textit{optional} since its default is \textbf{yes}.
In case backups must be turned off, the configuration should be the following:
\begin{lstlisting}[language=XML, caption=Turning off sequence numbers backup]
<backup>
    <execute>no</execute>
</backup>
\end{lstlisting}

\newparagraph For the other configuration options, two cases must be distinguished:

\paragraph{Backup to Blob} An example configuration would be the following one:
\begin{lstlisting}[language=XML, caption=Backup to blob]
<backup>
    <type>blob</type>
    <prefix>SeqNumber_</prefix>
</backup>
\end{lstlisting}

\noindent The text in the \texttt{<prefix>} tag will be prefixed to each blob created to avoid file name conflicts.
This prefix is  \textit{optional} field as default is set to \textbf{SeqNumber\textunderscore}.

\paragraph{Backup to file} An example configuration would be the following one:
\begin{lstlisting}[language=XML, caption=Backup to file]
<backup>
    <type>file</type>
    <directory>/backups</directory>
    <prefix>SeqNumber_</prefix>
</backup>
\end{lstlisting}

\noindent The \texttt{<directory>} tag is \textbf{mandatory} and must contain the path to the directory in which the backup files should be stored.

\noindent For the \texttt{<prefix>} tag the same considerations made for backup to blob applies.

\section{Use CPIM without migration system}
The \textit{migration.xml} file is not necessary if the user would not use the migration system. If the file is not present inside the \texttt{META-INF} folder, the NoSQL service will interact only with the underlying persistence provider without instantiating any of the classes required for the interaction with the migration system\textbf{•}.

\noindent This is possible due to the lazy initialization of those components that are initialized only the first time are actually used since are built with a singleton pattern.