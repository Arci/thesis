\section{Introduction}
In this appendix are described in detail the configurations that are available for the two developed Kundera extensions.
Are described the required properties that needs to be configured in the \textit{tersistence.xml} file and the available datastore specific properties that can be defined in the external datastore specific property file.

\section{Common configuration}
All the configuration is performed in the \textit{persistence.xml} file and so it follows the JPA standard.
The skeleton of the file is as follow:

\begin{verbatim}
<?xml version="1.0" encoding="UTF-8" standalone="no"?>
<persistence ... >
    <persistence-unit name="...">
    <provider>com.impetus.kundera.KunderaPersistence</provider>
        <class> ... </class>
        <exclude-unlisted-classes>true</exclude-unlisted-classes>
        <properties>
            <!-- kundera properties -->
        </properties>
    </persistence-unit>
</persistence>
\end{verbatim}

\noindent A name for the persistence unit is needed as it will be referenced inside the classes of the model.
In the \texttt{<calss>} tag must be listed, one per tag, the full name of the classes that needs to be handled through this persistence unit.
The only differences for the extensions are the kundera properties that needs to be specified inside the \texttt{<properties>} tag.

\section{GAE Datastore}
\label{appendix:datastore-config}
The configuration is done in the \textit{persistence.xml} file, two configuration are possible:
\begin{enumerate}
\item use the datastore instance within the appengine application
\item use a remote datastore instance through remote API
\end{enumerate}

\newparagraph The properties to be specified inside the \texttt{<properties>} tag for the first case are:
\begin{itemize}
\item \texttt{kundera.client.lookup.class} \textit{required}, must be set to \\ \texttt{it.polimi.kundera.client.datastore.DatastoreClientFactory}
\item \texttt{kundera.ddl.auto.prepare} \textit{optional}, possible values are:
\begin{itemize}
\item \texttt{create} which creates the schema (if not already exists)
\item \texttt{create-drop} which drop the schema (if exists) and creates it
\end{itemize}
\item \texttt{kundera.client.property} \textit{optional}, the name of the xml file containing the datastore specific properties.
\end{itemize}

\noindent In addition to the previous properties and in case of remote API, those properties are also necessary:
\begin{itemize}
\item \texttt{kundera.nodes} \textit{required}, url of the appengine application on which the datastore is located
\item \texttt{kundera.port} \textit{optional} default: 443, port used to connect to datastore
\item \texttt{kundera.username} \textit{required}, username of an admin on the remote server
\item \texttt{kundera.password} \textit{required}, password of an admin on the remote server
\end{itemize}

\noindent To test against local appengine runtime emulator the configuration is as follow:

\begin{verbatim}
<property name="kundera.nodes" value="localhost"/>
<property name="kundera.port" value="8888"/>
<property name="kundera.username" value="username"/>
<property name="kundera.password""/>
\end{verbatim}

\noindent the value for \texttt{kundera.password} does not matter.

\subsubsection{Datastore specific properties file}
A file with client specific properties can be created and placed inside the classpath, you need to specify its name in the persistence.xml file.
The skeleton of the file is the following:

\begin{verbatim}
<?xml version="1.0" encoding="UTF-8"?>
<clientProperties>
    <datastores>
        <dataStore>
            <name>datastore</name>
            <connection>
                <properties>
                    <property name="..." value="..."></property>
                </properties>
            </connection>
        </dataStore>
    </datastores>
</clientProperties>
\end{verbatim}

\noindent The available properties are:
\begin{itemize}
\item \texttt{datastore.policy.read} [eventual|strong] \textbf{default: strong}, set the read policy.
\item \texttt{datastore.deadline optional}, RPCs deadline in seconds.
\item \texttt{datastore.policy.transaction} [auto|none] \textbf{default: none}, define if use implicit transaction.

\end{itemize}

\section{Azure Table}
\label{appendix:table-config}
The configuration is done in the persistence.xml file, the properties to be specified inside the <properties> tag are:
\begin{itemize}
\item \texttt{kundera.username} \textit{required}, the storage account name (from azure portal)
\item \texttt{kundera.password} \textit{required}, the storage account key (from azure portal)
\item \texttt{kundera.client.lookup.class} \textit{required}, and set to\\\texttt{it.polimi.kundera.client.azuretable.AzureTableClientFactory}
\item \texttt{kundera.ddl.auto.prepare} \textit{optional}, possible values are:
\begin{itemize}
\item \texttt{create} which creates the schema (if not already exists)
\item \texttt{create-drop} which drop the schema (if exists) and creates it
\end{itemize}
\item \texttt{kundera.client.property} \textit{optional}, the name of the xml file containing the datastore specific properties.
\end{itemize}

\subsubsection{Datastore specific properties file}
A file with client specific properties can be created and placed inside the classpath, you need to specify its name in the persistence.xml file.
The skeleton of the file is the following:

\begin{verbatim}
<?xml version="1.0" encoding="UTF-8"?>
<clientProperties>
    <datastores>
        <dataStore>
            <name>azure-table</name>
            <connection>
                <properties>
                    <property name="..." value="..."></property>
                </properties>
            </connection>
        </dataStore>
    </datastores>
</clientProperties>
\end{verbatim}

\noindent The available properties are:
\begin{itemize}
\item \texttt{table.emulator} [true|false] \textbf{default: false}. If present (and set to true) storage emulator is used. When using dev server kundera.username and kundera.password in persistence xml are ignored.
\item \texttt{table.emulator.proxy} \textbf{default: localhost}. If storage emulator is used set the value for the emulator proxy.
\item \texttt{table.protocol} [http|https] \textbf{default: https}. Define the protocol to be used within requests.
\item \texttt{table.partition.default} \textbf{default: DEFAULT}.
The value for the default partition key, used when no one is specified by the user.
\end{itemize}