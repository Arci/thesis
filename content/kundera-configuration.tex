\section{Introduction}
In this appendix are described in detail the configurations available for the two developed Kundera extensions. Are described the required properties that needs to be configured in the \textit{persistence.xml} file and the available properties that can be defined in the external datastore specific properties file.

\section{Common configuration}
The main configuration is performed in the \textit{persistence.xml} file and it follows the JPA standard.
The template of the file is as follow:

\begin{lstlisting}[language=XML, caption=persistence.xml template]
<?xml version="1.0" encoding="UTF-8" standalone="no"?>
<persistence ... >
    <persistence-unit name="...">
    <provider>com.impetus.kundera.KunderaPersistence</provider>
        <class> ... </class>
        <exclude-unlisted-classes>true</exclude-unlisted-classes>
        <properties>
            <!-- kundera properties -->
        </properties>
    </persistence-unit>
</persistence>
\end{lstlisting}

\noindent A name for the persistence unit is mandatory as it will be referenced inside the classes of the model as shown in the snippet \ref{code:class-example} in which has been declared as schema the \texttt{kundera.keyspace} property to the persistence unit name.
The full name of the classes that needs to be handled through this persistence unit must be specified in the \texttt{<class>} tag.
Each extension needs different configuration that needs to be specified inside the \texttt{<properties>} tag.

\begin{lstlisting}[language=Java, caption=Declaring the schema, label=code:class-example]
@Table(schema = "gae-test@pu")
public class Employee {

    @Id
    @Column(name = "EMPLOYEE_ID")
    private String id;

    @Column(name = "NAME")
    private String name;

    @Column(name = "SALARY")
    private Long salary;
}
\end{lstlisting}

\section{GAE Datastore}
\label{appendix:datastore-config}
Two configuration are possible:
\begin{enumerate}
\item use the datastore instance within the app engine application;
\item use a remote datastore instance through remote API.
\end{enumerate}

\newparagraph The properties to be specified inside the \texttt{<properties>} tag for the first case are:
\begin{itemize}
\item \texttt{kundera.client.lookup.class} (\textit{required}), must be set to \\ \texttt{it.polimi.kundera.client.datastore.DatastoreClientFactory};
\item \texttt{kundera.ddl.auto.prepare} (\textit{optional}), possible values are:
\begin{itemize}
\item \texttt{create}, which creates the schema (if not already exists);
\item \texttt{create-drop}, which drop the schema (if exists) and creates it;
\end{itemize}
\item \texttt{kundera.client.property} (\textit{optional}), the name of the xml file containing the datastore specific properties.
\end{itemize}

\noindent In addition to the previous properties and in case of remote API, those properties are also necessary:
\begin{itemize}
\item \texttt{kundera.nodes} (\textit{required}), url of the app engine application on which the datastore is located;
\item \texttt{kundera.port} (\textit{optional}) default is \textbf{443}, port used to connect to datastore;
\item \texttt{kundera.username} (\textit{required}), username of an admin on the remote server;
\item \texttt{kundera.password} (\textit{required}), password of an admin on the remote server.
\end{itemize}

\noindent To test against local app engine run-time emulator the configuration is as follow:

\begin{lstlisting}[language=XML, caption=GAE Datastore emulator configuration]
<property name="kundera.nodes" value="localhost"/>
<property name="kundera.port" value="8888"/>
<property name="kundera.username" value="username"/>
<property name="kundera.password" value=""/>
\end{lstlisting}

\noindent in this case the value for \texttt{kundera.password} does not matter.

\subsubsection{Datastore specific properties file}
A file with client specific properties can be created and placed inside the classpath, its name must be specified in the \textit{persistence.xml} file through the property \texttt{<property name="kundera.client.property" value="filename.xml"/>}.
The template of the file is the following:

\begin{lstlisting}[language=XML, caption=GAE Datastore - datastore specific configuration]
<?xml version="1.0" encoding="UTF-8"?>
<clientProperties>
    <datastores>
        <dataStore>
            <name>datastore</name>
            <connection>
                <properties>
                    <property name="..." value="..."></property>
                </properties>
            </connection>
        </dataStore>
    </datastores>
</clientProperties>
\end{lstlisting}

\noindent The available properties are:
\begin{itemize}
\item \texttt{datastore.policy.read} (\texttt{optional}) [eventual\textbar strong] default is \textbf{strong}. Set the read policy;
\item \texttt{datastore.deadline} (\textit{optional}). RPCs deadline in seconds;
\item \texttt{datastore.policy.transaction} (\textit{optional}) [auto\textbar none] default is \textbf{none}. Define if use implicit transaction.
\end{itemize}

\section{Azure Table}
\label{appendix:table-config}
The properties to be specified inside the \texttt{<properties>} tag are:
\begin{itemize}
\item \texttt{kundera.username} (\textit{required}), the storage account name available from azure portal;
\item \texttt{kundera.password} (\textit{required}), the storage account key available from azure portal;
\item \texttt{kundera.client.lookup.class} (\textit{required}), must be set to\\\texttt{it.polimi.kundera.client.azuretable.AzureTableClientFactory};
\item \texttt{kundera.ddl.auto.prepare} (\textit{optional}), possible values are:
\begin{itemize}
\item \texttt{create}, which creates the schema (if not already exists);
\item \texttt{create-drop}, which drop the schema (if exists) and creates it.
\end{itemize}
\item \texttt{kundera.client.property} (\textit{optional}), the name of the xml file containing the datastore specific properties.
\end{itemize}

\subsubsection{Datastore specific properties file}
A file with client specific properties can be created and placed inside the classpath, its name must be specified in the \textit{persistence.xml} file.
The template of the file is the following:

\begin{lstlisting}[language=XML, caption=Azure Tables - datastore specific configuration]
<?xml version="1.0" encoding="UTF-8"?>
<clientProperties>
    <datastores>
        <dataStore>
            <name>azure-table</name>
            <connection>
                <properties>
                    <property name="..." value="..."></property>
                </properties>
            </connection>
        </dataStore>
    </datastores>
</clientProperties>
\end{lstlisting}

\noindent The available properties are:
\begin{itemize}
\item \texttt{table.emulator} (\textit{optional}) [true\textbar false] default is \textbf{false}. If present (and set to true) storage emulator is used. When using development server \texttt{kundera.username} and \texttt{kundera.password }in \textit{persistence.xml} are ignored;
\item \texttt{table.emulator.proxy} (\textit{optional}) default is \textbf{localhost}. If storage emulator is used set the value for the emulator proxy;
\item \texttt{table.protocol} (\textit{optional}) [http\textbar https] default is \textbf{https}. Define the protocol to be used within requests;
\item \texttt{table.partition.default} (\textit{optional}) default is \textbf{DEFAULT}.
The value for the default partition key, used when no one is specified by the user.
\end{itemize}