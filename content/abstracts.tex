%--------------------------------------------------------------------------------
% abstract in italian
%--------------------------------------------------------------------------------
\thispagestyle{empty}

\chapter*{Estratto}
Negli ultimi anni, specialmente per le applicazioni web, i requisiti sulla gestione dei dati sono drasticamente cambiati; le applicazioni devono gestire dati che per loro natura non sono strutturati e, specialmente, vengono generati in una quantit\`{a} tale che i sistemi tradizionali per la gestione dei dati non sono più sufficientemente performanti. Varie soluzioni sono state proposte come alternative ai classici DBMS; queste soluzioni prendono il nome di database NoSQL, per sottolineare il differente approccio adottato rispetto ai tradizionali DBMS.

\noindent Molte soluzioni NoSQL sono apparse in questi anni, e ognuna di esse utilizza un approccio differente nel cercare di soddisfare i requisiti sopra citati e molti altri. Queste tecnologie forniscono un insisme di API che mettono l'utente in condizioni di dover scrivere molto più codice rispetto a quanto sarebbe necessario utilizzando i tradizionali DBMS.

\noindent La mancanza di un linguaggio comune a tutti i database NoSQL richiede, quando si sviluppa un applicazione, una chiara e precisa conoscenza delle soluzioni disponibili sul mercato, per essere in grado di scegliere la tecnologia che pi\`{u} soddisfa i reauisiti dell'applicazione. Tuttavia, durante il ciclo di vita di un'applicazione, cambiare la soluzione NoSQL adottata, ad esempio a fronte di un cambiamento nei requisiti, pu\`{o} essere un problema. Questo problema \`{e} noto come \textit{vendor lock-in}.

\noindent Questo lavoro propone un modello che, usando la libreria CPIM e grazie alla ben nota interfaccia JPA, permetta all'utente di sviluppare applicazioni usando JPQL come linguaggio comune per molte tecnologie NoSQL e ottenere cos\`{i} una buona protabilità del codice, mitigando la complessit\`{a} delle tecnologie NoSQL senza perderne i vantaggi in termini di salabilit\`{a} e performance. Inoltre questo lavoro propone l'integrazione, all'interno della libreria CPIM, del sistema di migrazione e sincronizzazione \textit{Hegira}, per gestire la migrazione dei dati fra database NoSQL mitigando cos\`{i} il problema del vendor lock-in.

\cleardoublepage

%--------------------------------------------------------------------------------
% abstract in english
%--------------------------------------------------------------------------------
\thispagestyle{empty}

\chapter*{Abstract}
Within the last years, especially for web applications, data requirements have changed drastically; applications needs to handle information that are not always well structured and, more importantly, came in a volume that is not sustainable for traditional data management techniques. Solutions that tries to handle those new kind of data have emerged over the classical DBMS solutions; those solutions comes under the name of NoSQL databases, to underline the different approach they bring with respect to traditional  DBMS.

\noindent Many NoSQL technologies came into play in these years and each of them uses a different approach to handle all of the above requirements and many other. Those technologies provides a set of proprietary API that move toward the user a lot of programming effort with respect to DBMS solutions. 

\noindent The lack of a common language for NoSQL databases, require, when developing an application, a clear understanding of the available NoSQL solutions, to be able to choose the right technology for the application requirements. However, during the life cycle of the application, changing the adopted NoSQL technology, maybe due to requirements changes, may become a problem. This problem is known as \textit{vendor lock-in}.    

\noindent This work proposes a model that, using the CPIM library and through the well-known JPA interface, permits to the users to develop applications using JPQL (the JPA SQL-like query language) as a common language to many NoSQL technologies and thus achieve code probability, leveraging the complexity of NoSQL systems while exploiting the advantages that those technologies bring in terms of scalability and performance. Moreover this work proposes the integration, in the CPIM library, of \textit{Hegira} a migration and synchronization system to handle data migration among NoSQL database mitigating thus the vendor lock-in problem.

\cleardoublepage