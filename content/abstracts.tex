%--------------------------------------------------------------------------------
% abstract in italian
%--------------------------------------------------------------------------------
\thispagestyle{empty}

\chapter*{Estratto}
Negli ultimi anni, specialmente per le applicazioni web, i requisiti sulla gestione dei dati sono drasticamente cambiati; le applicazioni devono gestire dati che per loro natura non sono strutturati e, principalmente, vengono generati in una quantit\`{a} tale che i sistemi tradizionali per la gestione dei dati non sono più sufficientemente performanti. Varie soluzioni sono state proposte come alternative ai classici DBMS; queste soluzioni prendono il nome di NoSQL (Not Only SQL), per sottolineare il differente approccio adottato rispetto ai tradizionali DBMS.

\noindent Molti database NoSQL sono stati sviluppati in questi anni e, ognuno di essi, utilizza un approccio differente nel cercare di soddisfare i requisiti sopra citati.

\noindent Queste tecnologie forniscono un insisme di API che mettono l'utente in condizioni di dover scrivere molto pi\`{u}  codice rispetto a quanto sarebbe necessario utilizzando i tradizionali DBMS. 
La mancanza di un linguaggio comune a tutti i database NoSQL richiede una chiara e precisa conoscenza delle soluzioni disponibili sul mercato, per essere in grado di scegliere la tecnologia che pi\`{u} soddisfa i requisiti dell'applicazione. Tuttavia, durante il ciclo di vita di un'applicazione, cambiare la soluzione NoSQL adottata, ad esempio a fronte di un cambiamento nei requisiti o delle logiche di business, pu\`{o} essere un problema. Questo problema \`{e} noto come \textit{vendor lock-in}.

\noindent Questo lavoro propone un modello che, usando la libreria CPIM e sfruttando l'interfaccia JPA, permetta all'utente di sviluppare applicazioni usando Jun'interfaccia comune per molte tecnologie NoSQL e ottenere cos\`{i} una buona protabilit\`{a} del codice, mitigando la complessit\`{a} delle tecnologie NoSQL senza perderne i vantaggi in termini di salabilit\`{a} e performance. Inoltre questo lavoro propone l'integrazione, all'interno della libreria CPIM, del sistema di migrazione e sincronizzazione \textit{Hegira}, per gestire la migrazione e la sincronizzazione dei dati fra database NoSQL.

\cleardoublepage

%--------------------------------------------------------------------------------
% abstract in english
%--------------------------------------------------------------------------------
\thispagestyle{empty}

\chapter*{Abstract}
Within the last years, especially for web applications, data requirements have changed drastically; applications needs to handle information that are not always well structured and, more importantly, their volume is not sustainable for traditional data management techniques. Solutions that tries to handle those new kind of data have emerged over the classical DBMS solutions; those solutions comes under the name of NoSQL (Not Only SQL), to underline the different approach they bring with respect to traditional  DBMS.

\noindent Many NoSQL databases has been developed in these years and, each of them, uses a different approach to handle the previously mentioned requirements. Those technologies provides a set of proprietary API that move toward the user a lot of programming effort with respect to DBMS solutions. 

\noindent The lack of a common language for NoSQL databases, require a clear understanding of the available NoSQL solutions, to be able to choose the right technology for the application requirements. However, during the life cycle of the application, changing the adopted NoSQL technology, maybe due to changes in requirements or in business, may become a problem. This problem is known as \textit{vendor lock-in}.    

\noindent This work proposes a model that, using the CPIM library and through the JPA interface, permits to the users to develop applications using a common interface to interact with many NoSQL technologies and thus achieve code probability, leveraging the complexity of NoSQL systems while exploiting the advantages that those technologies bring in terms of scalability and performance. Moreover this work proposes the integration, in the CPIM library, of \textit{Hegira} a migration and synchronization system to handle data migration and synchronization among NoSQL database.

\cleardoublepage