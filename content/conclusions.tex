This work presented an approach that allow the users of the CPIM library to interact with different NoSQL solution through a common interface, identified in the JPA interface. This was possible by exploiting the functionalities of Kundera, a JPA compliant ORM built for NoSQL databases.

\noindent In the state of the art analysis of chapter \ref{chap:sota}, NoSQL database has been presented as an alternative to RBDMS and, the necessity of a common interface to communicate with different NoSQL solutions has been highlighted, while presenting the different solutions available in this direction.

\noindent Chapter \ref{chap:ps} provide a detailed description of the motivation that lead to the necessity of modifying the NoSQL service of the CPIM library and to the decision of integrating the migration and synchronization system \textit{Hegira} as part of the NoSQL service.

\noindent In Chapter \ref{chap:kundera} has been described what Kundera is, its architecture and, has been descried in detail, the development process that lead to the two new Kundera extensions, the first one to support Google Datastore and the second one to support Azure Tables. The extensions has been developed as part of a more general work on the CPIM library, indeed the develop of those Kundera extension was aimed to maintain the CPIM support for Google Datastore and Azure Tables. Chapter \ref{chap:cpim} described in detail the work made on the NoSQL service of CPIM, which is about the integration of Kundera as the unique persistence provider, and the integration of \textit{Hegira} to support transparent data synchronization and migration.

\noindent Finally, chapter \ref{chap:eval} shows the performance tests executed over the developed Kundera extensions with respect to the use of the low-level API, using the YCSB framework. The results showed that the overhead introduced by Kundera, and by the client extension, in terms of operation throughput and latency, is absolutely acceptable to justify the benefit that the use of Kundera introduces.
Furthermore, the chapter described \textit{Hegira-generator}, the application developed to generate data and testing the interaction with CPIM and \textit{Hegira}.

\newparagraph Possible future works should continue on both CPIM and Kundera. Indeed, CPIM needs to be updated to interacts with the latest version of the various cloud provider API and some components needs to be rewritten, as explained in section \ref{sec:cpim-problems} for the Queue service.

\noindent Further work can also be done in intercepting the user queries, that are then sent to the migration system, supporting for example the \textit{criteria API} discussed in section \ref{sec:cpim-intercept-queries}.

\noindent Finally some work can be done in adding to Kundera the support for more NoSQL databases such as Dynamo DB.
