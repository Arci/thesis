This work presented an approach of interacting with many different NoSQL databases through the CPIM library and the integration of the migration and synchronization system \textit{Hegira}.

\newparagraph The CPIM library has been modified in order to get rid of the previous implementation of the NoSQL service that was not able to guarantee full portability of the application code due to the numerous JPA implementation used to support different NoSQL databases. This work have made order in the CPIM library and added the ability for the users to interact with numerous different NoSQL solution through a common interface, identified in the JPA interface, thanks to Kundera, an open source JPA compliant ORM for NoSQL databases.

\noindent Furthermore have been produced two brand new clients for Kundera contributing thus to the project by adding the support for Google Datastore and Azure Tables. In chapter \ref{chap:eval}, the developed extensions has been tested in terms of throughput and latency in order to verify that the overhead added by Kundera and by its clients was not destructive for performance with respect to the use of low level API for interacting with the NoSQL databases.

\noindent The results showed that, since no significant overhead is added to the low level API version of NoSQLs, the approach we propose is worth the little loss of performance due to the benefit it brings in terms of code portability, through the CPIM library, and the ability of interacting with many different NoSQL databases with a unique and well known interface.

\newparagraph\newparagraph The NoSQL service of CPIM library has been further modified to integrate the required logic for interacting with the migration and synchronization system \textit{Hegira}, the work is described in chapter \ref{chap:cpim}. This mitigates the vendor lock-in problem by giving to the user the ability to change the adopted NoSQL technology and still be able to read the migrated data without the necessity of re-engineer the application.
 
\newparagraph Possible future works should continue on both CPIM and Kundera. Indeed, CPIM needs to be updated to interacts with the latest version of the various cloud provider API and some components needs to be rewritten, as explained in section \ref{sec:cpim-problems} for the Queue service.

\noindent Further work can also be done in intercepting the user queries, that are then sent to the migration system, supporting for example the \textit{criteria API} discussed in section \ref{sec:cpim-intercept-queries}.

\noindent Some work can be made in solving the problems that have prevented us from replicating the YCSB tests for Hbase. In this way the results presented in this work can be compared with the results of the tests performed by the Kundera team on the other clients.

\noindent Finally some work can be done in adding to Kundera the support for more NoSQL databases such as Dynamo DB.
