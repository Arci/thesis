Introduzione al lavoro. Inizia direttamente, senza nessuna sezione.

\section{\dots}
Argomenti trattati suddivisi sezione per sezione\dots

Per citare un articolo, ad esempio \cite{Ackley1987} o \cite{Ackley1987,Altenberg1994} utilizzare il comando \texttt{\\cite}. 

Per gestire i file di tipo \texttt{bib} esiste il programma \texttt{JabRef} disponibile sul sito \texttt{http://jabref.sourceforge.net/}.

\section*{Prima sezione}
\dots

\section*{Seconda sezione}
\dots

\section*{Outline of the Thesis}
This thesis is organized as follows: 
\begin{itemize}
\item In Chapter~\ref{chap:one} \dots
\item In Chapter~\ref{chap:two} \dots
\item In Chapter~\ref{chap:three} \dots
\item \dots
\end{itemize}
Finally, in Chapter~\ref{chap:conclusions}, \dots

\section*{Original Contributions}
This work include the following original contributions:
\begin{itemize}
\item \dots riassunto sintetico dei diversi contributi
\item \dots
\item \dots
\end{itemize}
